\documentclass[review]{elsarticle}

\makeatletter
\def\@author#1{\g@addto@macro\elsauthors{\normalsize%
    \def\baselinestretch{1}%
    \upshape\authorsep#1\unskip\textsuperscript{%
      \ifx\@fnmark\@empty\else\unskip\sep\@fnmark\let\sep=,\fi
      \ifx\@corref\@empty\else\unskip\sep\@corref\let\sep=,\fi
      }%
    \def\authorsep{\unskip,\space}%
    \global\let\@fnmark\@empty
    \global\let\@corref\@empty  %% Added
    \global\let\sep\@empty}%
    \@eadauthor={#1}
}
\makeatother

\usepackage{lineno}
\usepackage{color}
\usepackage[hidelinks]{hyperref}



\modulolinenumbers[5]

\journal{Journal of \LaTeX\ Templates}

%%%%%%%%%%%%%%%%%%%%%%%
%% Elsevier bibliography styles
%%%%%%%%%%%%%%%%%%%%%%%
%% To change the style, put a % in front of the second line of the current style and
%% remove the % from the second line of the style you would like to use.
%%%%%%%%%%%%%%%%%%%%%%%

%% Numbered
%\bibliographystyle{model1-num-names}

%% Numbered without titles
%\bibliographystyle{model1a-num-names}

%% Harvard
%\bibliographystyle{model2-names.bst}\biboptions{authoryear}

%% Vancouver numbered
%\usepackage{numcompress}\bibliographystyle{model3-num-names}

%% Vancouver name/year
%\usepackage{numcompress}\bibliographystyle{model4-names}\biboptions{authoryear}

%% APA style
%\bibliographystyle{model5-names}\biboptions{authoryear}

%% AMA style
%\usepackage{numcompress}\bibliographystyle{model6-num-names}

%% `Elsevier LaTeX' style
\bibliographystyle{elsarticle-num}
%%%%%%%%%%%%%%%%%%%%%%%

\begin{document}

\begin{frontmatter}

\title{Barreling formation during the axial compressive experiments on cubic
samples with orthotropic properties {\color{red} Preliminary title}}
%%\tnotetext[mytitlenote]{Fully documented templates are available in the
% elsarticle package on \href{http://www.ctan.org/tex-archive/macros/latex/contrib/elsarticle}{CTAN}.}

%% Group authors per affiliation:






%%\fntext[fn1]{This is the specimen author footnote.}
%%\fntext[fn2]{Another author footnote, but a little more longer.}
%%\fntext[fn3]{Yet another author footnote. Indeed, you can have any number of
% author footnotes.}

\author{Alexey Vorobyev\corref{cor1}}
\ead{alexey.vorobyev@angstrom.uu.se}

\cortext[cor1]{Corresponding author}

\author{Nico P. van Dijk}
\author{Kristofer Gamstedt}

\address{Uppsala University, Division of Appplied Mechanics,
Uppsala, Sweden }



\begin{abstract}
This template helps you to create a properly formatted \LaTeX\ manuscript.
\end{abstract}

\begin{keyword}
\texttt{elsarticle.cls}\sep \LaTeX\sep Elsevier \sep template
\MSC[2010] 00-01\sep  99-00
\end{keyword}

\end{frontmatter}

\linenumbers

\section{Introduction}
\begin{itemize}
\color{red}  
\item what problem was studied
\item Introduction of the problem
\item Give the context of the problem
\item Reviews of previous works
\item Justification of work
\item Scopes and objectives
\item Interest the reader
\item Provide the transition to the rest
\end{itemize}

\section{Materialss and Methods}
\section{Model}

The finite element program COMSOL Multiphysics 4.4 \cite{Comsol} was used for
the experimental simulations.
The entire cube with singular side was modeledwith 3D solid elements.
The platens of universal testing mashine were not considered in the model.
Instead the load were directly applied to the top surface of the cube. The
bottom surface of the cube was completely {\color{red}fastned}. However two
perpendicular edges of the bottom side were alloud to move in its perpendicular
direction. The same was done tot he top surface in addition to the applied
compressive force perpendicular to the surface.

{\color {red}Add the picture that is showing the axises and add the
description to the the text in accodance to that. Add the bottom surface, the
free adges the friction, the coeffitients, the variability of frictions}
\begin{itemize}
\color{red}

\item How was the rproblem studied 
\item Description of FEM model, parameters, application, details of the model
\item Describe the experimental setup, materials and procedure
\end{itemize}


\section{Results and Discussions}
\begin{itemize}
\color{red}
\item What were the findings?
\item Use diagrams and/or tables, objective and truthful presentation
\item Results and discussion distinguishable
\item New and old results and discussion
\end{itemize}

My name \cite{Dirac1953888} is Alexey. I am 29 \cite{Feynman1963118,
Dirac1953888} years old.

\section{Conclusions}

\section*{Acknowledgements}

\section*{References}

\bibliography{mybibfile}

\end{document}